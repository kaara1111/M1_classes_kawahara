理系学生の留学対して,様々な観点から解説されており非常にわかりやすかったです.
今まで,自分では留学といえば文系学生,特に外国語を専攻している人が行うものだと思っていました.
しかし,理系学生でも留学をすることで得られるものがあることを知り,留学に対するイメージが変わりました.
例えば,留学を通して得られるものとして,英語力の向上や留学先の文化を知ることができるということは勿論,
日本にいながらでは経験することは難しいような世界最先端の研究に触れるチャンスがあるということを知ることができました.
日本にも世界最先端の研究を行っているような研究者は多いと思いますが,海外の研究所に行くことで,
様々な価値観に触れながら様々な人々とディスカッションするということは,どうしても日本人学生が多くなってしまいがちな日本の研究室では経験することは難しいと思います.
また,シンポジウムでは留学することのメリットだけでなく,留学することで直面しうる困ったことについても実体験を交えて解説されており,
留学することのデメリットについても公平な目線から知ることができました.
今まで留学することは全く興味が無く,考えたことすらありませんでしたが,このシンポジウムを聞いた後,留学することを少しだけ考えるようになりました.
今まで完全に盲点だった留学について,今回のシンポジウムを聞いて少しでも知ることができたことは非常に良かったと思います.
ありがとうございました.